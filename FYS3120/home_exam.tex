\documentclass[12p,a4paper]{article}
\usepackage[utf8]{inputenc}
\usepackage[T1]{fontenc,url}
\usepackage{multicol}
\usepackage{multirow}
\usepackage{parskip}
\usepackage{lmodern}
\usepackage{microtype}
\usepackage{verbatim}
\usepackage{amsmath, amssymb}
\usepackage{tikz}
\usepackage{physics}
\usepackage{mathtools}
\usepackage{algorithm}
\usepackage{algpseudocode}
\usepackage{listings}
\usepackage{enumerate}
\usepackage{graphicx}
\usepackage{float}
\usepackage{hyperref}
\usepackage{tabularx}
\usepackage{siunitx}
\usepackage{fancyvrb}
\usepackage[makeroom]{cancel}
\usepackage[margin=2cm]{geometry}
\setlength\parindent{0pt}
\renewcommand{\baselinestretch}{1}

\newcommand{\half}{\frac{1}{2}}
\renewcommand{\b}{\boldsymbol}
\newcommand{\h}{\hat}
\newcommand{\m}{\mathbb}
\renewcommand{\exp}{e^}


\begin{document}

\title{FYS3110 -- Home Exam}
\author{
    \begin{tabular}{r l}
        Candidate Number 15229
    \end{tabular}}

\maketitle

\section*{Problem 1}
\subsection*{a)}
The pendulum oscilates with a velocity $v = b\dot{\phi}$. This gives a kinetic energi $K_1 = \half m v^2 = \half m b^2\dot{\phi}^2$.

In addition, we have the kinetic energy from the rotation of the wheel around it's own axis, given as $K_2 = \half I\omega^2 = \half I (\dot{\phi} + \alpha t)^2$. This gives a total kinetic energy of
\[
    K = K_1 + K_2 = \half m b^2\dot{\phi}^2 + \half I(\dot{\phi} + \alpha t)^2
\]

The potential energy is purely from gravity, given by the height of the mass mass point B.
\[
    V = -mgh = -mgb\cos{\phi}
\]

Giving a total lagrangian of 
\[
    L = K - V = \half m b^2\dot{\phi}^2 + \half I(\dot{\phi} + \alpha t)^2 + mgb\cos{\phi} \\ \\
\]
or, writing it out.
\begin{equation*}
    L = \half m b^2\dot{\phi}^2 + \half I \dot{\phi}^2 + I\alpha\dot{\phi}t + \half I \alpha^2 t^2 + mgb\cos{\phi} \\
\end{equation*}
\begin{equation}\label{eqn:Lagr}
    L = \half \qty(m b^2\dot + I)\dot{\phi}^2 + I\alpha\dot{\phi}t + mgb\cos{\phi} + \half I \alpha^2 t^2
\end{equation}

\subsection*{b)}
We have that
\[
    \pdv{L}{\phi} = -mgb\sin{\phi}
\]
\[
    \pdv{L}{\dot{\phi}} = \qty(mb^2 + I)\dot{\phi} + I \alpha t
\]
\[
    \dv{t}\pdv{L}{\dot{\phi}} = \qty(mb^2 + I)\ddot{\phi} + I\alpha
\]
Giving a lagrangian equation
\begin{align*}
    \dv{t}\pdv{L}{\dot{\phi}} - \pdv{L}{\phi} = 0 \\
    \qty(mb^2 + I)\ddot{\phi} + I\alpha + mgb\sin{\phi} = 0
\end{align*}


\subsection*{c)}
We have that
\begin{align*}
    \dv{f(\phi, t)}{t} &= \dv{t}\qty[I\alpha\phi t + \frac{1}{6} I \alpha^2t^3] \\
    &= I\alpha\dot{\phi}t + I\alpha\phi + \half I \alpha^2t^2
\end{align*}
We reconize the first and last term from the Lagrangian(\ref{eqn:Lagr}), and include the middle term by adding and subtracting it from the Lagrangian, giving
\begin{align*}
    L &= \qty[ \half \qty(m b^2\dot + I)\dot{\phi}^2 + mgb\cos{\phi} - I\alpha\phi ] + \qty[ I\alpha\dot{\phi}t + I\alpha\phi + \half I \alpha^2t^2 ] \\
    &= L' + \pdv{f(\phi, t)}{t}
\end{align*}
where
\begin{align*}
    L' = \half \qty(m b^2\dot + I)\dot{\phi}^2 + mgb\cos{\phi} - I\alpha\phi
\end{align*}


\subsection*{d)}
We have
\begin{align*}
    \pdv{L'}{\phi} = -mgb\sin{\phi} - I\alpha
\end{align*}
\begin{align*}
    \pdv{L'}{\dot{\phi}} = (mb^2 + I)\dot{\phi}
\end{align*}
\begin{align*}
    \dv{t}\pdv{L'}{\dot{\phi}} = (mb^2 + I)\ddot{\phi}
\end{align*}
giving a Lagrangian equation
\begin{align*}
    (mb^2 + I)\ddot{\phi} + mgb\sin{\phi} + I\alpha = 0
\end{align*}


\subsection*{e)}
The canonical momentum $p'_\phi$ of the coordinate $\phi$ is already calculated as
\begin{align*}
    p'_\phi = \pdv{L'}{\dot{\phi}} = (mb^2 + I)\dot{\phi}
\end{align*}
The Hamiltonian is defined as
\begin{align*}
    H &= \sum_i p_i \dot{q_i} - L = p'_\phi \dot{\phi} - L' \\
    &= (mb^2 + I)\dot{\phi}^2 - \half \qty(m b^2\dot + I)\dot{\phi}^2 - mgb\cos{\phi} + I\alpha\phi \\
    &= \half \qty(m b^2\dot + I)\dot{\phi}^2 - mgb\cos{\phi} + I\alpha\phi
\end{align*}


\subsection*{f)}
\begin{figure}
    \centering
    \includegraphics[width=0.49\textwidth]{{fig/lambd=0.2}.pdf}
    \includegraphics[width=0.49\textwidth]{{fig/lambd=0.4}.pdf}
    \includegraphics[width=0.49\textwidth]{{fig/lambd=0.6}.pdf}
    \includegraphics[width=0.49\textwidth]{{fig/lambd=0.8}.pdf}
    \includegraphics[width=0.49\textwidth]{{fig/lambd=1.0}.pdf}
    \includegraphics[width=0.49\textwidth]{{fig/lambd=1.2}.pdf}
    \includegraphics[width=0.49\textwidth]{{fig/lambd=1.4}.pdf}
    \includegraphics[width=0.49\textwidth]{{fig/lambd=1.6}.pdf}
    \caption{asdf}
\end{figure}

\subsection*{g)}
Firstly, we have
\begin{align*}
    p'_\phi = (mb^2 + I)\dot{\phi} \quad\quad \Rightarrow \quad\quad
    \dot{\phi} = \frac{1}{mb^2 + I}p'_\phi
\end{align*}
giving the Hamiltonian as a function of $\phi$ and $p'_\phi$ only:
\begin{align*}
    H &= \half \qty(m b^2\dot + I)\dot{\phi}^2 - mgb\cos{\phi} + I\alpha\phi \\
    &= \half \qty(m b^2\dot + I)\frac{1}{(mb^2 + I)^2}{p'_\phi}^2 - mgb\cos{\phi} + I\alpha\phi \\
    &= \half \frac{{p'_\phi}^2}{mb^2 + I} - mgb\cos{\phi} + mgb\lambda\phi \\
\end{align*}



\section*{Oppgave 2}
\subsection*{a)}


\subsection*{b)}
Conservation of energy gives that particle A and a must have a combined energy equal to that of particle B before the decacy.
\begin{align*}
    E_B = E_A + E_a\\
    \sqrt{P_B^2c^2 + m_B^2c^4} = \sqrt{P_A^2c^2 + m_A^2c^4} + \sqrt{P_a^2c^2 + m_a^2c^4}
\end{align*}
In the rest frame of A, $P_A = 0$, and conservation of momentum gives that $P_a^2 = P_B^2$ (total momentum before decay must equal total momentum after dacay). This gives
\begin{align*}
    \sqrt{P_a^2 + m_B^2c^2} = \sqrt{m_A^2c^2} + \sqrt{P_a^2 + m_a^2c^2} \\
    \qty(\sqrt{P_a^2 + m_B^2c^2} - \sqrt{P_a^2 + m_a^2c^2})^2 = m_A^2c^2 \\
    P_a^2 + m_B^2c^2 - \sqrt{P_a^2 + m_B^2c^2}\sqrt{P_a^2 + m_a^2c^2} + P_a^2 + m_a^2c^2 = m_A^2c^2 \\
    P_a^2 + m_B^2c^2 - \sqrt{P_a^4 + P_a^2m_B^2c^2 + P_a^2m_a^2c^2 + m_a^4c^4} + P_a^2 + m_a^2c^2 = m_A^2c^2 \\
\end{align*}

    


\end{document}
