\documentclass[10pt,a4paper]{article}
\usepackage[utf8]{inputenc}
\usepackage[T1]{fontenc,url}
\usepackage{parskip}
\usepackage{lmodern}
\usepackage{microtype}
\usepackage{verbatim}
\usepackage{amsmath, amssymb}
\usepackage{tikz}
\usepackage{physics}
\usepackage{mathtools}
\usepackage{algorithm}
\usepackage{algpseudocode}
\usepackage{listings}
\usepackage{enumerate}
\usepackage{enumitem}
\usepackage{graphicx}
\usepackage{float}
\usepackage{hyperref}
\usepackage{tabularx}
\usepackage{siunitx}
\usepackage{multicol}
\usepackage{multirow}
\usepackage{framed}
\usepackage{fancyvrb}
\usepackage{xcolor}
\usepackage[a4paper, margin=0.2cm]{geometry}

\renewcommand{\baselinestretch}{1}
\renewcommand{\b}{\textbf}
\renewcommand{\exp}{e^}

\newcommand{\infint}{\int_{-\infty}^{\infty}}
\newcommand{\zeroinfint}{\int_{-\infty}^{\infty}}

\newcommand{\infsum}{\sum_{n=-\infty}^{\infty}}
\newcommand{\zeroinfsum}{\sum_{n=0}^{\infty}}
\newcommand{\oneinfsum}{\sum_{n=1}^{\infty}}

\newcommand{\holine}{\line(1,0){280}}

\newcommand{\half}{\frac{1}{2}}

\newcommand{\gr}{\colorbox{green}}
\newcommand{\yl}{\colorbox{yellow}}

\setlength{\columnsep}{0.4cm}
\setlength{\columnseprule}{0.06cm}
\setlength{\FrameSep}{0.2cm}





\begin{document}
\begin{multicols}{2}



\section*{Ordinary Differential Equations}

% \begin{tikzpicture}[level distance=2.5cm,
%   level 1/.style={sibling distance=8cm},
%   level 2/.style={sibling distance=6cm},
%   level 3/.style={sibling distance=3cm}]
%     \node[circle, draw](z){$ODEs$}
%         child{node[circle,draw]{1st Order}
%             child{node[circle,draw]{Separable}}
%             child{node[circle,draw]{Linear}}
%             }
%         child{node[circle,draw]{2nd Order}
%             child{node[circle,draw]{Homogeneous}
%                 child{node[circle,draw]{Constant coeff.}}
%                 child{node[circle,draw]{Euler-Cauchy}}
%                 }
%             child{node[circle,draw]{Inhomogeneous}
%                 child{node[circle,draw]{Constant coeff.}}
%                 child{node[circle,draw]{Factorization}}
%                 child{node[circle,draw]{Variation of parameters}}
%                 }
%             };
% \end{tikzpicture}



\subsection*{First Order, Linear, ODEs - Integrating Factor}
\[
    y'(x) + P(x) y(x) = Q(x)
\]

\[
    y(x)\mu(x) = \int Q(x)\mu(x) \dd{x} + C \quad\quad\text{with}\quad\quad \mu(x) = \exp{\int P(x)\dd{x}}
\]


\[
    \int (uv') = uv - \int(u'v)
\]



\subsection*{Second Order, Homogenous, Linear ODEs, with constant coefficients - Particular Equation}
\[
    y''(x) + ay'(x) + by(x) = 0
\]

Solve the particular equation
\[
    \lambda^2 + a\lambda + b = 0
\]
for $\lambda_1$ and $\lambda_2$.


\subsubsection*{Two, real roots}
\[
    y(x) = C_1\exp{\lambda_1 x} + C_2\exp{\lambda_2 x}
\]


\subsubsection*{One, real root}
\[
    y(x) = (C_1 + xC_2)\exp{\lambda x}
\]


\subsubsection*{Two, complex roots}
\begin{equation*}
\begin{split}
    y(x) = A\exp{\lambda_1 x} + B\exp{\lambda_2 x} = \exp{-a/2 x}\qty[A\exp{i\omega x} + B\exp{-i\omega x}] =\\
    \exp{-a/2 x}\qty[\hat{A}\cos{\omega x} + \hat{B}\sin{\omega x}]
\end{split}
\end{equation*}






\newpage


\section*{Fourier Series}
\[
    f(x) = \half a_0 + \oneinfsum a_n\cos(\frac{n\pi x}{L}) + \oneinfsum a_n\sin(\frac{n\pi x}{L})
\]
\[
    a_n = \frac{1}{L}\int\limits_{-L}^{L}f(x) \cos(\frac{n\pi x}{L}) \dd{x} \quad\quad
    b_n = \frac{1}{L}\int\limits_{-L}^{L}f(x) \sin(\frac{n\pi x}{L}) \dd{x}
\]
\[
    f(x) = \sum_{n=-\infty}^{\infty} c_n \exp{in\pi x/L}  \quad\quad
    c_n = \frac{1}{2L}\int_{-L}^L f(x) \exp{-in\pi x/L}
\]

\subsection*{Even and Odd functions}
If $f(x)$ is \b{even}:
\[
    a_n = \frac{2}{L}\int_0^L f(x)\cos(\frac{n\pi x}{L}) \dd{x} \quad\quad b_n = 0
\]
If $f(x)$ is \b{odd}:
\[
    a_n = 0 \quad\quad b_n = \frac{2}{L}\int_0^L f(x)\sin(\frac{n\pi x}{L}) \dd{x}
\]

\holine

\section*{Fourier Transforms}
\[
    f(x) = \frac{1}{\sqrt{2\pi}} \infint F(k) \exp{ikx} \dd{k} \quad\quad
    F(k) = \frac{1}{\sqrt{2\pi}} \infint f(x) \exp{-ikx} \dd{x}
\]

\subsection*{Odd and even functions}
If $f(x)$ is an odd function, $f(x) = -f(-x)$, the Fourier transform can be done using only sine (as cosine is symmetric around 0):
\[
    f(x) = \sqrt{\frac{2}{\pi}} i\int\limits_0^{\infty} F(k) \sin(k x) \dd{k}   \quad
    F(k) = \sqrt{\frac{2}{\pi}} i\int\limits_0^{\infty} f(x) \sin(k x) \dd{x}
\]
If $f(x)$ is even, $f(x) = f(-x)$, we need only cosine (as sine is anti-symmentric):
\[
    f(x) = \sqrt{\frac{2}{\pi}} \int\limits_0^{\infty} F(k) \cos(k x) \dd{k}   \quad
    F(k) = \sqrt{\frac{2}{\pi}} \int\limits_0^{\infty} f(x) \cos(k x) \dd{x}
\]


\subsection*{FT of a derivative}
\[
    \mathcal{F}\qty[f^{(n)}(x)] = (ik)^n \mathcal{F}\qty[f(x)]
\]

\newpage


\section*{Partial Differential Equations}
\subsection*{Wave Equation}
\[
    \pdv[2]{y}{x} = \frac{1}{c^2}\pdv[2]{y}{t}
\]
Solution as linear combination of the the solutions
\[
    y(x,t) = \qty{\begin{matrix} \sin(kx) \\ \cos(kx) \end{matrix}} \times \qty{\begin{matrix} \sin(kvt) \\ \cos(kvt) \end{matrix}}
\]
The end-points are usually fixed at 0, leaving only the $\sin(kt)$ term, and forcing $k=n\pi/L$.

If the velocity is 0 at $t=0$, we discard the sin-velocity term and get

\[
    y(x,t) = \oneinfsum b_n\sin(\frac{n\pi x}{L})\cos(\frac{n\pi v t}{L})
\]

If position is 0 at $t=0$, we discard the cos-velocity term instead.

Initial position will be on the form
\[
    y(x,0) = \oneinfsum b_n \sin(\frac{n\pi x}{L}) = f(x)
\]
where $f(x)$ is the initial position or velocity. The coefficients $b_n$ are now Fourier coefficients, given as:
\[
    b_n = \frac{1}{2L}\int_{0}^{L} f(x)\sin(\frac{n\pi x}{L}) \dd{x}
\]

\newpage





\end{multicols}
\end{document}
