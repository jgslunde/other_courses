\documentclass[12p,a4paper]{article}
\usepackage[utf8]{inputenc}
\usepackage[T1]{fontenc,url}
\usepackage{multicol}
\usepackage{multirow}
\usepackage{parskip}
\usepackage{lmodern}
\usepackage{microtype}
\usepackage{verbatim}
\usepackage{amsmath, amssymb}
\usepackage{tikz}
\usepackage{physics}
\usepackage{mathtools}
\usepackage{algorithm}
\usepackage{algpseudocode}
\usepackage{listings}
\usepackage{enumerate}
\usepackage{graphicx}
\usepackage{float}
\usepackage{hyperref}
\usepackage{tabularx}
\usepackage{siunitx}
\usepackage{fancyvrb}
\usepackage[makeroom]{cancel}
\usepackage[margin=2.4cm]{geometry}
\renewcommand{\baselinestretch}{1}
\renewcommand{\exp}{e^}
\renewcommand{\b}{\boldsymbol}
\newcommand{\h}{\hat}
\newcommand{\m}{\mathbb}
\newcommand{\half}{\frac{1}{2}}
\renewcommand{\exp}{e^}
\setlength\parindent{0pt}


\begin{document}
\title{STK1110 -- Oblig 1}
\author{
    \begin{tabular}{r l}
        Jonas Gahr Sturtzel Lunde & (\texttt{jonassl})
    \end{tabular}}
% \date{}    % if commented out, the date is set to the current date

\maketitle

\hspace{10cm}

\section*{Oppgave 2}
\subsection*{a)}
Ettersom ligning (1) fra oppgaven er en t-fordeling med $n-1$ frihetsgrader, vet vi at den følger
\begin{equation}
    P\qty(t_{\alpha/2,\, n-1} < \frac{\bar{X} - \mu}{S/\sqrt{n}} < t_{1-\alpha/2,\, n-1}) = 1-\alpha
\end{equation}
der $t_{\alpha/2,\, n-1}$ og $t_{1-\alpha/2,\, n-1}$ er $\alpha/2$ og $1-\alpha/2$ persentilene til en t-fordeling med $n-1$ frihetsgrader. 

Løser vi ulikheten inni parantesen for $\mu$ får vi at
\begin{equation*}
    \bar{X} - t_{\alpha/2,\, n-1}\cdot\frac{S}{\sqrt{n}} < \mu < \bar{X} + t_{1 - \alpha/2, n-1}\cdot\frac{S}{\sqrt{n}}
\end{equation*}
som er $100(1-\alpha)\%$ konfidensintervallet til $\mu$.


\subsection*{b}
Ettersom ligning (1) fra oppgaven er kjikvadrat-fordelt med $n-1$ frihetsgrader, vet vi at den tilfredsstiller
\begin{equation}
    P\qty(\chi_{\alpha/2,\, n-1} < \frac{(n-1)}{\sigma^2}S^2 < \chi_{\alpha/2,\, n-1}) = 1 - \alpha
\end{equation}
der $\chi_{\alpha/2,\, n-1}$ og $\chi_{1-\alpha/2,\, n-1}$ er $\alpha/2$ og $1-\alpha/2$ persentilene til en kjikvadrat-fordeling med $n-1$ frihetsgrader. 

Løser vi ulikheten inni parantesen for $\sigma$ får vi at
\begin{equation}
    \sqrt{\frac{(n-1)}{\chi_{\alpha/2,\, n-1}}}S < \sigma < \sqrt{\frac{(n-1)}{\chi_{1-\alpha/2,\, n-1}}}S
\end{equation}




\section*{Oppgave 3}
\subsection*{a)}
\begin{equation*}
    F(x) = \int\limits_{-\infty}^x f(x) \dd{x} = \int\limits_{\kappa}^x \theta\kappa^\theta x^{-\theta-1} \dd{x} = \qty[\theta\kappa^\theta\frac{x^{-\theta}}{-\theta}]_\kappa^\theta = 1 - \qty(\frac{\kappa}{x})^\theta
\end{equation*}

\begin{equation*}
    F(x) = 1 - \qty(\frac{\kappa}{x})^\theta = \half \ \Rightarrow \ \frac{\kappa}{x} = \frac{1}{2}
\end{equation*}


\subsection*{b)}
\begin{equation*}
    E(X) = \int\limits_{-\infty}^\infty x\cdot f(x) \dd{x}
         = \int\limits_{\kappa}^\infty \theta\kappa^\theta x^{-\theta} \dd{x}
         = \qty[ \theta\kappa^\theta \frac{x^{-\theta+1}}{-\theta+1} ]_{\kappa}^\infty 
         = 0 - \theta\kappa^\theta \frac{\kappa^{-\theta + 1}}{-\theta +1}
         = \frac{\theta\kappa}{\theta - 1}
\end{equation*}


\subsection*{c)}
Vi omskriver den gitte definisjonen til å gi $X$ som definisjon av $Y$:
\begin{align*}
    Y = 2\theta \qty[\ln(X) - \ln(\kappa)] &= 2\theta\ln(X/\kappa) \\
    \exp{Y/2\theta} &= X/\kappa \\
    X &= \kappa \exp{Y/2\theta}
\end{align*}
Vi setter dette inn i uttrykket vårt for den kummulative fordelingsfunksjonen:
\begin{align*}
    F(Y) = 1 - \qty(\frac{\kappa}{\kappa \exp{y/2\theta}})^\theta = 1 - \exp{-y/2}
\end{align*}
\begin{align*}
    f(y) = F'(y) = \half\exp{-y/2} = \frac{1}{2^{2/2}\Gamma(2/2)}x^{2/2-1}\exp{-x/2}
\end{align*}
som vi ser er en kjikvadratfordeling med 2 frihetsgrader.


\subsection*{d)}
\begin{align*}
    E(X) = \frac{\theta\kappa}{\theta - 1} = \bar{X} \ \Rightarrow \ 
    \theta\kappa = \theta\bar{X} - \bar{X} \ \Rightarrow \ 
    \theta = \frac{\bar{X}}{\bar{X} - \kappa}
\end{align*}
Momentestimatoren til $\theta$ er altså $\hat{\theta} = \frac{\bar{X}}{\bar{X} - \kappa}$.


\subsection*{e)}
Sannsynligheten for at kombinasjonen av tilfeldige variable $X_1, X_2,...,X_n$ i $n$ uavhengige forsøk blir $x_1, x_2,...,x_n$ vil være produktet av de individuelle sannsynlighetene
\begin{align*}
    f(x_1, x_2, ..., x_n; \theta) = \prod_{i=1}^n \theta\kappa^\theta\qty(\frac{1}{x_i})^{\theta+1} = \theta^n\kappa^n\theta \qty(\prod_{i=1}^n (x_i)^{-1})^{\theta+1}
\end{align*}

Vi skal finne maks-verdien til denne fordelingen. Vi tar først logaritmen av fordelingen, ettersom den deler toppunkt med sin logaritme, og dette er enklere å regne med.
\begin{align*}
    \ln[f(x_1, x_2, ..., x_n; \theta)]
    &= \ln(\theta^n) + \ln(\kappa^n\theta) + \ln\qty[\qty(\prod_{i=1}^n x_i^{-1})^{\theta+1}] \\
    &= n\ln(\theta) + n\theta\ln(\kappa) + (\theta + 1)\qty[ \sum_{i=1}^n -\ln(x_i) ]
\end{align*}
Deriverer, setter lik 0, og løser for $\theta$:
\begin{align*}
    \dv{\theta} \ln[f(x_1, x_2, ..., x_n; \theta)] &= \frac{n}{\theta} + n\ln(\kappa) - \sum_{i=1}^n \ln(x_i) = 0 \\
    n &= \theta\qty[\sum_{i=1}^n \ln(x_i) - n\ln(\kappa)] \\
    \hat{\theta} &= \frac{n}{\sum_{i=1}^n \ln(x_i) - n\ln(\kappa)}
\end{align*}
som da er maximum likelihood estimatoren for $\theta$.


\subsection*{f)}
Vi har en ny stokastisk variabel
\begin{align*}
    Y = 2n\frac{\theta}{\hat{\theta}} = 2\theta\qty[\sum_{i=1}^n \ln(x_i) - n\ln(\kappa)] = \sum_{i=1}^n 2\theta\qty[\ln(x_i) - \ln(\kappa)] = \sum_{i=1}^n Z
\end{align*}
hvor $Z \sim \chi_2^2$, altså $Z$ er kjikvadratfordelt med 2 frihetsgrader.

Fra s.316 i læreboka har vi at summen av kjikvadratfordelinger selv er en kjikvadratfordeling, med ny frihetsgrad lik summen av frihetsgradene, som betyr at $Z$ er en kjikvadratfordeling med $2n$ frihetsgrader:
\begin{align*}
    \sum_{i=1}^n Z \sim \chi_{2n}^2
\end{align*}

\end{document}
