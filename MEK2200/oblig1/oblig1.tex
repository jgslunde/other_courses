\documentclass[12p,a4paper]{article}
\usepackage[utf8]{inputenc}
\usepackage[T1]{fontenc,url}
\usepackage{multicol}
\usepackage{multirow}
\usepackage{parskip}
\usepackage{lmodern}
\usepackage{microtype}
\usepackage{verbatim}
\usepackage{amsmath, amssymb}
\usepackage{tikz}
\usepackage{physics}
\usepackage{mathtools}
\usepackage{algorithm}
\usepackage{algpseudocode}
\usepackage{listings}
\usepackage{enumerate}
\usepackage{graphicx}
\usepackage{float}
\usepackage{hyperref}
\usepackage{tabularx}
\usepackage{siunitx}
\usepackage{fancyvrb}
\usepackage[makeroom]{cancel}
\usepackage[margin=2cm]{geometry}
\renewcommand{\baselinestretch}{1}
\renewcommand{\b}{\boldsymbol}
\newcommand{\h}{\hat}
\newcommand{\m}{\mathbb}
\newcommand{\xvec}{\begin{pmatrix} x\\y\\z \end{pmatrix}}
\newcommand{\ptensor}{\begin{pmatrix} a & b & 0 \\ b & d & 0 \\ 0 & 0 & e \end{pmatrix}}
\newcommand{\half}{\frac{1}{2}}
\setlength\parindent{0pt}


\begin{document}
\title{MEK2200 -- Oblig 1}
\author{
    \begin{tabular}{r l}
        Jonas Gahr Sturtzel Lunde & (\texttt{jonassl})
    \end{tabular}}
% \date{}    % if commented out, the date is set to the current date

\maketitle

\hspace{10cm}


\section*{Oppgave 1}

\begin{table}[H]
    \centering
    \caption{Multiprogram sets}
    \label{multiprogram}
    \begin{tabular}{|c|c|c|c|c|}
        \hline
        &&&&\\
                      & \bf Vektornotasjon          & \bf Indeksnotasjon     & \bf Dyadisk notasjon    & \bf Komponentform \\
        &&&&\\
        \hline
        &&&&\\
        Ytreprodukt  &  $\m {C} = \b a \otimes \b b$  &  $C_{ij} = a_i b_j$  &  $\m C = a_i b_j(\b i_i \otimes \b i_j)$  &  \\
        &&&&\\
        \hline
        &&&&\\
        Divergens  &  $c = \nabla \cdot \b a$  & c = $\frac{\partial a_i}{\partial x_i}$  &  &  c = $(\frac{\partial a_1}{\partial x_1} + \frac{\partial a_2}{\partial x_2} + ...)$ \\
        &&&&\\
        \hline
        &&&&\\
        Curl  &  $\b c = \nabla \times \b a$  & $c_i = \epsilon_{ijk} \frac{\partial a_k}{\partial x_j}$   &  $\b c = \frac{\partial}{\partial x_i}a_j(\b i_i \times \b i_j)$ & \\
        &&&&\\
        \hline
        &&&&\\
        Kryssprodukt  &  $\b c = \b a \times \b b$  &   $\epsilon_{ijk}a_jb_k$ &  $\b c = a_i b_j (\b i_i \times \b i_j)$  &   \\
        &&&&\\
        \hline
        &&&&\\
        Gradient  &  $\m C = \nabla \b a$  &  $C_{ij} = \frac{\partial a_i}{\partial x_j}$  &  $\m C = \frac{\partial}{\partial x_i}a_j(\b i_i \otimes \b i_j)$ & \\
        &&&&\\
        \hline
    \end{tabular}
\end{table}


\section*{Oppgave 2}
\subsection*{a)}
Cauchy's andre spenningsrelasjoner sier at spenningstensoren $\m P$ er symetrisk. Dette innebærer at $c = b$ i vårt tilfelle. Vi kan da skrive $\m P$ som
\begin{align*}
\m P = \begin{pmatrix} a & b & 0 \\ b & d & 0 \\ 0 & 0 & e \end{pmatrix}
\end{align*}

\subsection*{b)}
Spenningsvektoren på et plan med normalvektor $\b n$ er definert som
\begin{align*}
    \b P_n = \b n \cdot \m P = \qty[\frac{1}{\sqrt{2}},\ \frac{1}{\sqrt{2}},\ 0] \cdot \begin{pmatrix} a & b & 0 \\ b & d & 0 \\ 0 & 0 & e \end{pmatrix} = \qty[\frac{a + b}{\sqrt{2}},\ \frac{b + d}{\sqrt{2}},\ 0]
\end{align*}

\subsection*{c)}
De normale og tangentielle og komponentene av spenningsvektoren er henholdsvis definert som
\begin{align*}
    P_{nn} = \b P_n \cdot \b n = \qty[\frac{a + b}{\sqrt{2}},\ \frac{b + d}{\sqrt{2}},\ 0] \cdot \qty[\frac{1}{\sqrt{2}},\ \frac{1}{\sqrt{2}},\ 0] = b + a/2 + d/2
\end{align*}

\begin{align*}
    P_{nt} = |\b P_n \times \b n| = \qty| \qty[0,\ 0,\ a/2 + b/2 - (b/2 + d/2)] | = a/2 - d/2
\end{align*}


\subsection*{d)}
Vi vet at prinsipalretningene og prinsipalspenningene korresponderer til egenvektorene og egenverdiene til spenningstensoren.
Egenverdiene til en matrise finnes ved den karakterisiske ligningen
\begin{align*}
    |\m P - I \lambda| = 0
\end{align*}
For vår stresstensor har vi at
\begin{align*}
    |\m P - I \lambda| &= \begin{vmatrix} a - \lambda & b & 0 \\ b & d - \lambda & 0 \\ 0 & 0 & e - \lambda \end{vmatrix}
    = (e-\lambda)\begin{vmatrix} a - \lambda & b \\ b & d-\lambda \end{vmatrix} \\ \\
    &= (e-\lambda)\qty[(a-\lambda)(d-\lambda) - (b)(b)] \\
\end{align*}
som har løsningen
\begin{align*}
    \lambda_1 = e \\
\end{align*}
og ligningen
\begin{align*}
    (a-\lambda)(d-\lambda) - (b)(b) = 0 \\
    \lambda^2 - (a + d)\lambda + ad - b^2 = 0 \\
\end{align*}
som har løsningene
\begin{align*}
    \lambda_2 = \frac{1}{2}\qty[\sqrt{a^2 -2ad +4b^2 +d^2} +a +d] \\
    \lambda_3 = \frac{1}{2}\qty[-\sqrt{a^2 -2ad +4b^2 +d^2} +a +d]
\end{align*}

Egenvektorene er definert som
\begin{align*}
    \m P \b x_i = \lambda_i \b x_i
\end{align*}
Som gir ligningene
\begin{align*}
    \ptensor \xvec_1 &= e \xvec_1 \\
    \ptensor \xvec_2 &= \frac{1}{2}\qty[\sqrt{a^2 -2ad +4b^2 +d^2} +a +d] \xvec_2 \\
    \ptensor \xvec_3 &= \frac{1}{2}\qty[-\sqrt{a^2 -2ad +4b^2 +d^2} +a +d] \xvec_3
\end{align*}
med løsninger
\begin{align*}
    \b x_1 &= \qty[0,\ 0,\ 1] \\
    \b x_2 &= \qty[ -\frac{-a+d-\sqrt{a^2 +4b^2 -2ad +d^2}}{2b},\ 1,\ 0] \\
    \b x_3 &= \qty[ -\frac{-a+d+\sqrt{a^2 +4b^2 -2ad +d^2}}{2b},\ 1,\ 0]
\end{align*}
Dette er altså prinsipalspenningene og retningene.




\section*{Oppgave 3}
\subsection*{a)}
\begin{figure}[H]
    \centering
    \includegraphics[width=0.7\textwidth]{fig/task3_field}
\end{figure}
\begin{figure}[H]
    \centering
    \includegraphics[width=0.7\textwidth]{fig/task3_square.pdf}
\end{figure}
\begin{figure}[H]
    \centering
    \includegraphics[width=0.4\textwidth]{{fig/task3_a=0.2}.pdf}
    \includegraphics[width=0.4\textwidth]{{fig/task3_a=0.5}.pdf}
    \includegraphics[width=0.4\textwidth]{{fig/task3_a=0.8}.pdf}
    \includegraphics[width=0.4\textwidth]{fig/task3_a=1.pdf}
\end{figure}

Ettersom vi ser fra figurene at kvadratet holder en rektangulær form, kan vi regne arealet som produktet av to sidekanter, som vi finner som avstanden mellom to punkter:
\begin{align*}
    |(1,0) - (0,1)|\cdot|(1,0) - (0,-1)| = 2
\end{align*}
Vi regner ut hvor disse 3 punktene befinner seg etter forskyvningen.
\begin{align*}
    (1 + \alpha y, 0 + \alpha x) = (1, \alpha) \\
    (0 + \alpha y, 1 + \alpha x) = (\alpha, 1) \\
    (0 + \alpha y, -1 + \alpha x) = (-\alpha, -1)
\end{align*}

Arealet blir da
\begin{align*}
    &|(1, \alpha) - (\alpha, 1)| \cdot |(1, \alpha) - (-\alpha, -1)| = |(1-\alpha,\alpha-1)|\cdot |(1 + \alpha, \alpha+1)| \\ \\
    &= \sqrt{(1-\alpha)^2 + (\alpha-1)^2}\sqrt{(\alpha+1)^2 + (\alpha+1)^2} \\ \\
    &= \sqrt{2(1-\alpha)^2}\sqrt{2(\alpha+1)^2} = 2(1 - \alpha^2)
\end{align*}
Vi ser at arealet av firkanten er 2 ved $\alpha=0$, og 0 ved $\alpha = 1$, slik vi forventet fra figurene.

\subsection*{b)}
Forskyvningsfeiltet $\b u = [\alpha y, \alpha x]$ bestemmer forskyvningen til et punkt i feltet. Forskyvningsforskjellen mellom to punkt i feltet blir da
\begin{align*}
    \Delta\b u = [\alpha\Delta y,\ \alpha\Delta x]
\end{align*}

Tensoren for relative forskyvningsforskjeller er (i to dimensjoner) gitt som
\begin{align*}
    \m D =
    \begin{pmatrix}
        \frac{\partial u_1}{\partial x_1} & \frac{\partial u_1}{\partial x_2} \\[0.5em]
        \frac{\partial u_2}{\partial x_1} & \frac{\partial u_2}{\partial x_2}
    \end{pmatrix}
    =
    \begin{pmatrix}
        0 & \alpha \\
        \alpha & 0
    \end{pmatrix}
\end{align*}


\section*{Oppgave 4}
\subsection*{a)}

Vi har spenningstensoren definert som
\begin{align}\label{eqn:Pij}
    P_{ij} = \lambda\nabla\cdot \b u \delta_{ij} +2\mu \epsilon_{ij} = 2\mu \epsilon_{ij}
\end{align}

Vi finner tøyningstensoren for systemet vårt, definert som
\begin{align*}
    \epsilon_{ij} = \half\qty(\frac{\partial u_i}{\partial x_j} + \frac{\partial u_j}{\partial x_i})
\end{align*}
\begin{align*}
    \m \epsilon =
    \begin{pmatrix}
        \frac{\partial u_1}{\partial x_1} & \half\qty(\frac{\partial u_1}{\partial x_2} + \frac{\partial u_2}{\partial x_1}) & \half\qty(\frac{\partial u_1}{\partial x_3} + \frac{\partial u_3}{\partial x_1}) \\[0.5em]
        \half\qty(\frac{\partial u_2}{\partial x_1} + \frac{\partial u_1}{\partial x_2}) & \frac{\partial u_2}{\partial x_2} & \half\qty(\frac{\partial u_2}{\partial x_3} + \frac{\partial u_3}{\partial x_2}) \\[0.5em]
        \half\qty(\frac{\partial u_3}{\partial x_1} + \frac{\partial u_1}{\partial x_3}) & \half\qty(\frac{\partial u_2}{\partial x_3} + \frac{\partial u_3}{\partial x_2}) & \frac{\partial u_3}{\partial x_3} \\
    \end{pmatrix}
    =
    \begin{pmatrix}
        0  &  -\half qz  &  \half qy \\[0.5em]
        -\half qz  &  0  &  0 \\[0.5em]
        \half qy  &  0  &  0
    \end{pmatrix}
\end{align*}
Som insatt i \ref{eqn:Pij} gir spenningsmatrisen
\begin{align*}
    \m P = \mu q
    \begin{pmatrix}
        0  &  -z  &  y \\
        -z  &  0  &  0 \\
        y  &  0  &  0
    \end{pmatrix}
\end{align*}


\subsection*{b)}
For å finne prinsipalspenninger og retninger sitter vi igjen med et egenverdiproblem. Vi skal løse ligningen
\begin{align*}
    \m P\b x &= \lambda\b x \\ \\
    \begin{pmatrix}
        0  &  -z  &  y \\
        -z  &  0  &  0 \\
        y  &  0  &  0
    \end{pmatrix}
    \xvec &= \lambda\xvec
\end{align*}
Jeg orker ikke løse enda et egenverdiproblem for hånd, så WolframAlpha skal få lov til å ta denne. Prinsipalspenningene og retningene i kartesiske koordinater er
\begin{align*}
    \b x_1 = \qty[0,\ \frac{y}{z},\ 0]& \ \ \ \ \lambda_1 = 0 \\
    \b x_2 = \qty[-\frac{\sqrt{y^2+z^2}}{y},\ -\frac{z}{y},\ 1]& \ \ \ \ \lambda_2 = -\sqrt{y^2+z^2} \\
    \b x_3 = \qty[\frac{\sqrt{y^2+z^2}}{y},\ -\frac{z}{y},\ 1]& \ \ \ \ \lambda_3 = \sqrt{y^2+z^2}
\end{align*}
\end{document}
