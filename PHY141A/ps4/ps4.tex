\documentclass[12p,a4paper]{article}
\usepackage[utf8]{inputenc}
\usepackage[T1]{fontenc,url}
\usepackage{multicol}
\usepackage{multirow}
\usepackage{parskip}
\usepackage{lmodern}
\usepackage{microtype}
\usepackage{verbatim}
\usepackage{amsmath, amssymb}
\usepackage{tikz}
\usepackage{physics}
\usepackage{mathtools}
\usepackage{algorithm}
\usepackage{algpseudocode}
\usepackage{listings}
\usepackage{enumerate}
\usepackage{graphicx}
\usepackage{float}
\usepackage{hyperref}
\usepackage{tabularx}
\usepackage{siunitx}
\usepackage{fancyvrb}
\usepackage[makeroom]{cancel}
\usepackage[margin=2.0cm]{geometry}
\renewcommand{\baselinestretch}{1}
\renewcommand{\exp}{e^}
\renewcommand{\b}[1]{\boldsymbol{\mathrm{#1}}}
\newcommand{\h}{\hat}
\newcommand{\m}{\mathbb}
\newcommand{\half}{\frac{1}{2}}
\renewcommand{\exp}{e^}
\renewcommand{\bar}{\overline}
\setlength\parindent{0pt}


\begin{document}
\title{PHYSICS 141A -- Problem Set 4}
\author{
    \begin{tabular}{r l}
        Jonas Gahr Sturtzel Lunde & (\texttt{jonassl})
    \end{tabular}}
% \date{}    % if commented out, the date is set to the current date

\maketitle

\hspace{10cm}

\section*{Exercise 1 - Simon 9.4}
We have defined our vibrational nodes as
\begin{align}\label{eqn:1}
    \delta x_n = A\exp{i\omega t}\exp{-ikna}
\end{align}
while we have the dispersion relation
\[
    \omega(k) = \sqrt{\frac{\kappa}{m}} \qty|\, \sin(\frac{ka}{2})| = \omega_{max} \qty|\, \sin(\frac{ka}{2})|
\]
We observe that if we set $\omega$ larger than $\omega_{max}$, we will obtain a complex $k$, as sin doesn't output values larger than 1 for real arguments. Let us set $\omega = \sigma\omega_{max}$, $\sigma > 1$, such that
\begin{align}\label{eqn:2}
    \sigma = \frac{\omega}{\omega_{max}} = \qty|\, \sin(\frac{ka}{2})|
\end{align}

Now, considering a complex $k = k_r + i k_c$, we can write equation \ref{eqn:1} as
\[
    \delta x_n = A\exp{i\omega t} \exp{-ik_rna}\exp{k_cna}
\]
The last term, containing $k_c$, will either blow up as for large $n$, if $k_c > 0$, or disappear for large $n$, if $k_c < 0$. 

We can write this as a decay-relation
\[
    \delta x_n = C(t, k) \exp{-qa}, \quad q = -k_ca
\]
where $C(t, k)$ is some harmonic planewave solution.

Writing out \ref{eqn:2} using the $\sin(x) = \frac{i}{2}\qty(\exp{-ix} - \exp{ix})$, we get
\[
    \sigma = \sin(\frac{ka}{2}) = \frac{i}{2}\qty(\exp{-ika/2} - \exp{ika/2})
\]
Multiplying each side by $2i\exp{ika/2}$, we get
\[
    2i\sigma\exp{ika/2} =  \exp{ika} - 1
\]
which can be written as 
\[
    x^2 - 2i\sigma x - 1 = 0, \quad\quad  x = \exp{ika/2}
\]
which is a 2nd degree polynomial eqn with solutions
\[
    \exp{ika/2} = x = \sigma i \pm \sqrt{1- \sigma^2}
\]
Now, we have defined that $\sigma > 1$, such that $\sigma^2 > 1$, and we can define $\sqrt{1-\sigma^2} = \gamma i$, where $\gamma$ must be real. This gives
\begin{align*}
    \exp{ika/2} &= \sigma i \pm \gamma i \\
    \frac{ika}{2} &= \ln(\sigma i \pm \gamma i) = \ln(i) + \ln(\sigma \pm \gamma) = \frac{\pi}{2}i + \ln(\sigma \pm \gamma) \\
    k &= \frac{\pi}{a} + \frac{2}{a}i \ln(\sigma \pm \gamma) = k_r + i k_c
\end{align*}
where we have that $k_c = \ln(\sigma \pm \gamma)$, as we know both $\sigma$ and $\gamma$ must be real.
Inserting gives
\[
    k_c = \ln(\sigma \pm \frac{\sqrt{1-\sigma^2}}{i})
\]
which is real. Remember that $\sigma = \omega/\omega_{max}$.


\section*{Exercise 2 - Simon 9.6}
Given the mode
\begin{align}\label{eqn:3}
    \delta x_n = A\exp{i\omega t}\exp{q|n|a}
\end{align}
We have the double time derivative:
\[
    \ddot{\delta x_n} = A\omega^2\exp{i\omega t}\exp{q|n|a}
\]
From Simon eqn (9.1), we have Newtons equation of motion written out for neighboring nodes:
\[
    m \ddot{\delta x_n} = \kappa\qty(\delta x_{n+1} + \delta x_{n-1} - 2\delta x_n)
\]
Inserting for equation \ref{eqn:3} on the left hand side, and the derivative on the right, we get
\begin{align*}
    Am\omega^2\exp{i\omega t}\exp{q|n|a} &= \kappa A\exp{i\omega t}\qty[\exp{q|n+1|a} + \exp{iq|n-1|a} - 2\exp{q|n|a}] \\
    \omega^2m &= \kappa\qty[\exp{qa(|n+1|-|n|)} + \exp{qa(|n-1|-|n|)} - 2\exp{qa(|n|-|n|)}] 
\end{align*}
Now, for $n \geq 1$, we have $|n| = n$, $|n-1| = n-1$ and $|n+1| = n+1$, giving
\[
    \omega^2m = \kappa\qty[\exp{qa} + \exp{-qa} - 2], \quad\quad n \geq 1
\]
For $n \leq -1$, we have $|n| = -n$, $|n-1| = -n+1$, and $|n+1| = -n-1$, giving
\[
    \omega^2m = \kappa\qty[\exp{-qa} + \exp{qa} - 2], \quad\quad n \leq -1
\]
which turns out to be the same thing.
At $n=0$, we end up getting the same story. This could also have been seen due to symetry, as swithcing $n$-direction only switches $|n-1|$ and $|n+1|$, leaving the same expression.

We regocnize this as
\[
    m\omega^2 =2\kappa\qty[1 - \cos(qa)]
\]
which we regocnize from Simon, with $q = k$. This has frequencies
\[
    \omega = \sqrt{\frac{\kappa}{m}}\qty|\, \sin(\frac{qa}{2})|
\]

In 9.4, we solved for $k$, given $\omega > \omega_{max}$, allowing for a complex $k$. Since $q$ is supposed to be complex, we apply this solution, giving
\[
    q = \frac{\pi}{a} + \frac{2}{a}i\ln(\frac{\omega}{\omega_{max}} \pm \frac{\sqrt{1 - \omega^2/\omega_{max}^2}}{i})
\]

We know from 9.4 that these kind of waves fall off for $\omega > \omega_{max}$, which is kindoff what we would expect from the waves of the impurity, if they are not in ressonanse with the frequencies of the other material, and has a higher frequency.

\end{document}