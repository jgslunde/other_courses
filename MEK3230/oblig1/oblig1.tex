\documentclass[12p,a4paper]{article}
\usepackage[utf8]{inputenc}
\usepackage[T1]{fontenc,url}
\usepackage{multicol}
\usepackage{multirow}
\usepackage{parskip}
\usepackage{lmodern}
\usepackage{microtype}
\usepackage{verbatim}
\usepackage{amsmath, amssymb}
\usepackage{tikz}
\usepackage{physics}
\usepackage{mathtools}
\usepackage{algorithm}
\usepackage{algpseudocode}
\usepackage{listings}
\usepackage{enumerate}
\usepackage{graphicx}
\usepackage{float}
\usepackage{hyperref}
\usepackage{tabularx}
\usepackage{siunitx}
\usepackage{fancyvrb}
\usepackage[makeroom]{cancel}
\usepackage[margin=2cm]{geometry}
\renewcommand{\baselinestretch}{1}
\renewcommand{\b}{\boldsymbol}
\newcommand{\h}{\hat}
\newcommand{\m}{\mathbb}
\newcommand{\xvec}{\begin{pmatrix} x\\y\\z \end{pmatrix}}
\newcommand{\ptensor}{\begin{pmatrix} a & b & 0 \\ b & d & 0 \\ 0 & 0 & e \end{pmatrix}}
\newcommand{\half}{\frac{1}{2}}
\setlength\parindent{0pt}


\begin{document}
\section{Korte Oppgaver}
\subsection{Vannstråle}
\subsubsection{}
Bernoulli gjelder ikke på kryss av strømlinjer. Det er heller ingen bakgrunn for å påstå at hastigheten er høyere på innsiden av strålen.


\subsubsection{}



\subsection{Atmosfærestrømning}



\section{Lange oppgaver}
\subsection{Rankine-virvel - en forenklet tornado}
Vi ser på en inkompressibel, friksjonsfri strømning gitt i sylinderkoordinater som
\begin{align*}
  u_\theta &= \begin{cases}
    \Gamma r \qquad &r < a \\
    \frac{\Gamma a^2}{r} &r > 0
  \end{cases}\\
  u_r &= 0 \\
  u_z & = 0
\end{align*}

Fordi vi kan stryke relativt mange ledd fra Navier Stokes, setter vi den opp på vektor-form, og ikke i sin forferdelige sylinderform:
\begin{align*}
    \frac{\partial \b u}{\partial t} + \b u \nabla \b u = - \frac{\nabla P}{\rho} + \nu \nabla^2 \b u + \b f_e
\end{align*}






\begin{table}[H]
    \centering
    \begin{tabular}{|c|c|c|}
        \hline
        &&\\
        \bf Parametere          & \bf Enheter     & \bf Dimmensjoner \\
        &&\\
        \hline
        &&\\
        $\rho$  &  $kg/m^3$  &  $[M][L^{-1}]$ \\
        &&\\
        \hline
        &&\\
        $\frac{D\b u}{D t}$  &  $m/s^2$  &  $[L][T^{-2}]$ \\
        &&\\
        \hline
        &&\\
        $\b \Omega$  &  $1/s$  &  $[T^{-1}]$ \\
        &&\\
        \hline
        &&\\
        $\b u$  &  $m/s$  &  $[L][T^{-1}]$ \\
        &&\\
        \hline
        &&\\
        $\b r$  &  $m$  &  $[L]$ \\
        &&\\
        \hline
        &&\\
        $\nabla p$  &  $kg/s^2m^2$  &  $[M][T^{-2}][L^{-1}]$ \\
        &&\\
        \hline
        &&\\
        $\mu$  &  $kg/sm$  &  $[M][T^{-1}][L^{-1}]$ \\
        &&\\
        \hline
    \end{tabular}
\end{table}

\end{document}
